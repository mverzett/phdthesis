\chapter*{Abstract}

This thesis presents the results of the search for the standard model (SM) process $pp \To \W\H \To \ell \nu \tau_\ell \tau_h$, where the Higgs boson decays to tau leptons and one of them decays hadronically. The presence of the additional lepton from the decay of the W boson largely reduces the backgrounds in this channel.

Originally predicted by P. Higgs, F. Englert et al. in 1964, the Higgs boson has been for long time the only missing particle predicted by standard model and has been therefore the target of major efforts in experimental particle physics. Searches for this elusive particles have been conducted without any success both at LEP and Tevatron. The Large Hadron Collider (LHC) and two major experiments (ATLAS and CMS) have been built with the main purpose of casting some light in this longstanding mystery. 

These efforts culminated on July 4$^{th}$ 2012, when both the experiments announced the observation of a resonance at 125 GeV in the $\gamma \gamma$ and ZZ spectra combined. While this announcement generated a well deserved enthusiasm, fermionic decays of this resonance were still to be observed. The search for the decay process $\rm{H} \To \tau \tau$ has been the most promising channel to first observe the decay of the SM Higgs boson to fermions as the other main competitor, $\rm{H} \To b\bar{b}$, has larger branching fraction but much higher backgrounds. 

In this work we focus on the associated production of a Higgs boson and a W boson as a low--background environment to observe such decay. This final state is part of a wider $\rm{H} \To \tau \tau$ search program performed by CMS which is briefly outlined in the first chapter of this work. 

The main source of background in this final state comes from a wide range of processes (mainly top quark production and W boson production in association with jets) in which at least one of the leptons in the final state comes from a misidentified quark or gluon jet. To model this source of backgrounds a fully data--driven method %that exploits a k--Nearest Neighbors classifier
is developed.

The search excludes the presence of a SM Higgs boson with more than 2.5 times its expected cross section. The observed exclusion limit is consistent within $2\sigma$ from the  background--only hypothesis. This results has been combined with the other searches for Higgs decays into tau pairs performed at the CMS experiment, leading to the first observation of such process.