\chapter*{Abstract}

This thesis presents the results of the search for the Standard Model process $\W\H \To \ell \tau_\ell \tau_h$, where the Higgs boson decays to tau leptons and one of them decays hadronically. The presence of the additional lepton form the decay of the W boson largely reduces the backgrounds in this channel.

The Higgs boson has been for long time the only missing particle predicted by Standard Model and has been therefore the target of major efforts in the experimental particle physics filed. Searches for this elusive particles have been conducted without any success both at LEP and Tevatron as a side project in their respective physics programme. The LHC and two of its major experiments (ATLAS and CMS) have been built with the main purpose of casting some light in this longstanding mystery. 

These efforts culminated on the 4$^{th}$ of july 2012, when both the experiments announced the observation of a resonance at 125 GeV in the $\gamma \gamma$ and ZZ spectra combined. While this announcement triggered a well deserved enthusiasm, the sign of the fermionic decays of this resonance were still to be observed. The search for the decay process $\rm{H} \To \tau \tau$ has been the most promising channel to first observe the decay of the SM Higgs boson to fermions. 

In this work we focus on the associated production of a Higgs boson and a W boson as a low--background environment to observe such decay. This final state is part of a wider $\rm{H} \To \tau \tau$ search program performed by CMS which is briefly outlined in the first chapter of this work. 

The main source of background in this peculiar final state comes from a wide range of processes (mainly $t\bar{t}$ and W plus jets) in which at least one of the leptons in the final state comes from a mis--identified quark or gluon jet. To model this source of backgrounds a fully data--driven method that exploits a k--Nearest Neighbors classifier is developed.

The final results do not show any presence of the Higgs boson, but rather an under--fluctuation of the backgrounds. Given the results an upper limit on the presence of the Higgs boson is set as a function of the Higgs mass.