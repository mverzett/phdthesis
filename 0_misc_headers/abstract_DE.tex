\chapter*{Abstract}

Diese Dissertation stellt die Ergebnisse der Suche nach dem Standardmodell-Prozess $pp \To \W\H \To \ell \nu \tau_\ell \tau_h$, in dem das Higgs-Boson in Tau-Leptonen zerf\"allt, wobei eines der Tau-Leptonen hadronisch zerf\"allt, dar. Die Anwesenheit eines zus\"atzlichen Leptons aus dem W-Boson-Zerfall f\"uhrt zu einer starken Reduzierung des Untergrunds in diesem Kanal.

Das Higgs-Boson, welches urspr\"unglich von P. Higgs, F. Englert et al. 1964 vorhergesagt wurde, war lange Zeit das einzig fehlende, vom Standardmodell (SM) vorhergesagte Teilchen und war daher Kern vieler gro\ss{}er Anstrengungen in der experimentellen Teilchenphysik. Suchen nach diesem schwer zu entdeckendem Teilchen am LEP und auch am Tevatron waren erfolglos. Der Large Hadron Collider (LHC) und die beiden Hauptexperimente (ATLAS und CMS) wurden haupts\"achlich gebaut, um Fortschritte in der Suche nach dem Higgs-Boson zu erzielen.

Die Anstrengungen machten sich am 4. Juli 2012 bezahlt, als beide Experimente die Entdeckung einer Resonanz um 125 GeV in den Spektren der $\gamma \gamma$ und der ZZ Endzust\"ande bekannt gaben. Obwohl diese Nachricht enthusiastisch aufgenommen wurde, mussten fermionische Zerf\"alle dieser Resonanz erst noch entdeckt werden. Die Suche nach dem Zerfallsprozess $\rm{H} \To \tau \tau$ war der vielversprechendste Kanal um diesen Zerfall zu entdecken und stand im direkten Wettbewerb mit dem Zerfall $\rm{H} \To b\bar{b}$, der zwar ein gr\"o\ss{}eres Verzweigungsverh\"altnis besitzt, aber daf\"ur auch mit gr\"o\ss{}erem Untergrund zu k\"ampfen hat.

In dieser Arbeit konzentrieren wir uns auf die assoziierte Produktion eines Higgs-Bosons und eines W-Bosons, um m\"oglichst kleine Untergrundprozesse zu haben. Dieser Endzustand ist Teil eines gr\"o\ss{}eren Suchprogramms des CMS-Experiments, welches im ersten Kapitel dieser Arbeit kurz beschrieben wird.

Der Hauptuntergrund in diesem Endzustand setzt sich aus einer Reihe von Prozessen (haupts\"achlich Top-Quark- und W-Boson-Produktion zusammen mit Jets) zusammen, in denen mindestens eines der Leptonen im Endzustand aus der Fehlidentifizierung eines Quark- oder Gluonjets stammt. Um diese Untergrundquelle zu modellieren wird eine datengetriebene Methode entwickelt, die sogenannte k--Nearest Neighbors Classifier verwendet.

In dieser Suche wurde der Produktionswirkungsquerschnitt des Higgs-Bosons mit einer kombinierten Signalst\"arke von $\mu = -2.6\pm1.7$ gemessen, die unterhalb der SM-Erwartung liegt, jedoch mit dieser innerhalb eines Signifikanzniveaus von $2.1\,\sigma$ kompatibel ist.
Das Ergebnis dieser Analyse wurde mit anderen Suchen nach Tau-Lepton-Zerf\"allen des Higgs-Bosons im CMS-Experiment kombiniert und resultierte damit in der ersten Beobachtung dieses Prozesses.
%Die Suche schlie\ss{}t die Existenz eines SM-Higgs-Bosons mit einem Produktionswirkungsquerschnitt gr\o\ss{}er als 2,5-mal der SM-Erwartung aus. Die beobachtete Ausschlussgrenze ist innerhalb von $2\sigma$ konsistent mit der reinen Untergrundhypothese. Dieses Ergebnis wurde mit den anderen Suchen nach dem Higgs-Boson im Zerfall in Tau-Paare am CMS-Experiment kombiniert, welches letztendlich zur Beobachtung dieses Prozesses f\"uhrte.